\chapter{Maintenance and Repair}

Maintenance work must follow all safety instructions in {\ThisBook} strictly.

Maintenance, calibration, and repair work must be performed in accordance with the latest {\ThatBook}, which can be viewed on Lebai's official website:\url{https://i.lebai.ltd/d/s/}.

% \vfill

\info{\begin{itemize}
	\item Keep the product clean (the robot should be cleaned every day).
	\item Check for loose fasteners every three months.
\end{itemize}}

Repairs must be performed by authorized service providers or professional technicians designated by the company. It is necessary to ensure the required safety level for maintenance and repair work, comply with applicable national or regional work safety regulations, and check whether all safety functions work normally.

The purpose of maintenance work is to ensure the normal operation of the system, or to help it recover when the system fails. Maintenance includes fault diagnosis and actual maintenance. Maintenance includes fault diagnosis and actual maintenance.

% \newpage

The following safety procedures and warnings must be followed when operating the robot or control box:

\innerinfo{
\includegraphics[width=1.5cm]{image/ic_w_flash.pdf}
\\
\includegraphics[width=1.5cm]{image/ic_w_warning.pdf}
}{}{\small
\begin{itemize}
\item It is necessary to take necessary precautions during maintenance to prevent others from reconnecting the power to the system during maintenance. After power off, check the system again to make sure it is indeed powered off.
\item Check the ground connection of the power plug before restarting the system.
\item It is strictly prohibited for users to disassemble the robot or control box solely by themselves.
\item Lebai-designated service providers or professional technicians must comply with ESD (electrostatic discharge) regulations during maintenance work.
\item Prevent water or dust from entering the robot or control box.
% \end{itemize}}

% \vfill

% \innerinfo{\includegraphics[width=1.5cm]{image/ic_w_warning.pdf}
% \\\includegraphics[width=1.5cm]{image/ic_w_flash.pdf}
% }{}{\begin{itemize}
\item It is forbidden to change any information in the software security configuration. If the safety parameters change, the entire robot system should be regarded as a new one, which means that all safety audit processes, such as risk assessment, must be updated.
\item Replace faulty parts with new parts with the same part number or equivalent parts approved by our company.
\item Reactivate all disabled safety measures immediately after completion of the work.
\item Record all maintenance operations and save them in the technical documents relevant to the entire robot system.
\item The control box has no parts that can be repaired solely by the end user. If maintenance or repair service is required, please contact your dealer or our company.
\end{itemize}
}
