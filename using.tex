\chapter{使用机器人}

场景为操作和使用机器人的基本功能模块,编写一个新场景即可开始操作机器人。在讲述如何创建和编写一个场景之前,我们先介绍下乐白机器人的场景编辑器以及如何进行拖动示教 操作。
场景编辑器
编辑器类型
乐白机器人场景编辑操作支持使用如下两种编辑器进行编辑:
时间轴编辑器 
(需补充分区示意图)






图 4 1 时间轴编辑器功能分区
 可视化编辑器,低门槛无需理解任何逻辑关系,只需要按照操作说明和引导完成相应的动作块场景编写,即可让机器人运动起来。
代码编辑器

 
图 4 2  代码编辑器功能分区
(设计改一下lebai乐白为L MASTER)
针对高级用户,我们提供更专业的基于Lua语言的代码编辑器,使用代码编辑器可以完成各种复杂的逻辑和自定义的操作流程,可定制化程度高,需要具备一定的编程基础和逻辑基础。
切换编辑器类型
代码编辑器仅在专家模式下可用,从时间轴编辑器转换到代码编辑器的步骤如下:
	进入“设置”->“操作模式“页或点击时间轴编辑器编辑区右上角“时间轴编辑器”右边的 按钮,进入操作模式选择页,选择专家模式,点击操作模式页右上角的”保存“按钮;
	返回之前编辑的场景;
	进入时间轴编辑器,点击编辑区右上角的“时间轴编辑器”按钮,在下拉弹框中选择“代码编辑器”进行切换
 
图 4 3 切换编辑器类型

 	注意:
当某个场景的编辑器类型从时间轴编辑器切换成代码编辑器后,无法再将当前场景的编辑操作转换为时间轴编辑器。

类型图标说明
 在场景列表和场景编辑器页面出现此图标表示该场景为时间轴编辑器类型;
 在场景列表和场景编辑器页面出现此图标表示该场景为代码编辑器类型。

拖动示教
在讲述如何添加位置动作块之前,有必要先了解下乐白机器人的拖动示教方法。拖动示教有如下两种操作方法,可任选其一:
在L MASTER点击示教图标 使机器人进入示教状态,拖动机器人到达指定位置,再次点击示教图标,结束示教状态;
长按机器人末端凸按钮,使机器人进入示教状态,拖动机器人到达指定位置后,释放机器人末端凸按钮,结束示教状态。

 	警告:
在使用拖动示教前,请务必确保正确设置好末端设备的质量与质心,不正确的设置可能导致误伤。 

 	警告:
示教过程中注意关节角度的转动不可超出安全范围,否则可能导致机器人急停或其他故障。 

编写场景
创建新场景
点击L MASTER首页的场景按钮,进入场景列表,点击该页面工具栏右侧的“添加场景”按钮,输入场景名称,完成新场景的创建。
 
图 4 4  创建场景
使用时间轴编辑器编辑
时间轴编辑器支持的动作块类型有:
位置
手爪
等待
消息提示
数字I/O
模拟I/O
信号量
负载配置
点击时间轴编辑器任意蓝色 号图标,可以弹出选择添加动作块对话框,如图 4 5 添加动作块对话框所示。
 
图 4 5 添加动作块对话框
下面章节依次讲述不同动作块类型的使用方法。
位置动作块
位置控制作为机器人控制里面的核心控制模块,在时间轴编辑器中,只需要按照如下教程添加位置动作块,即可让机器人按照预定的位置动作进行相应类型的轨迹运动。
添加位置 
在时间轴编辑器中添加位置有如下两种操作方法,可任选其一:
时间轴编辑器内编辑区蓝色加号(补图标)按钮,选择位置图标,在打开的添加位置对话框(如图 4 6 添加位置对话框所示)中输入位置名称,并选择蓝色“添加”字样,即可将机器人当前位置保存在一个新的位置动作块 ;
双击机器人末端凸按钮,系统将机器人当前位置保存在一个新的位置动作块。
 
图 4 6  添加位置对话框
通过结合拖动示教和上述添加位置动作块的方法,按照您的需求添加想要的位置动作块(可任意多个),直至场景编写完成。
编辑位置
将光标放在位置块上使当前位置块处于焦点状态,如图 4 7  位置块焦点状态所示:
 
图 4 7  位置块焦点状态
轨迹类型切换区的两个按钮用于切换关节空间的运动(movej)和笛卡尔空间的直线运动(movel)。
 表示关节空间 的运动;
 表示笛卡尔空间 的直线运动。
编辑操作按钮区从左至右的按钮依次为:
(每个按钮补充对应的图标,同时不带背景,上面两个轨迹类型的图标同样需更新)
样式:可以更换位置块背景颜色;
微调:对当前位置块存储的位置数据进行微调及更新;
速度及加速时间:调整位置块的速度和加速时间 ;
复制:复制一个当前的动作块;
删除:可删除当前动作块。
	微调
在场景编辑时,动作轨迹中的位置需要精准细微调整,可以对单个位置进行微调。选择微调图标,页面可自动跳转至微调页面。
微调页面中虚拟机器人和真实机器人底座重叠展示,其中虚拟机器人表示当前位置块存储的目标位置,真实机器人表示机器人当前位置。通过实际位置和目标位置切换按钮 进行切换,可查看目标位置与实际位置的数值。
	坐标空间微调
 
图 4 8  坐标空间微调
(添加标注)
坐标空间可选择相对于底座/末端
相对于底座时,参考坐标系可使用直角坐标系或圆柱坐标系;
相对于末端时,参考坐标系仅支持直角坐标系。
 
图 4 9  坐标空间示意图
参考坐标系选择“相对于底座”,就是以机器人底座平面的圆心作为世界坐标系原点,通过选择直角坐标系为调整方式,在微调页面位置信息展示框中输入位置(X, Y, Z)或姿态 (Rz, Ry, Rx)的数值,机器人会自动移动到您所需要的位置;或通过拉动屏幕下方各个方向的拉杆进行操作,当释放拉杆时,微调停止。点击“更新”按钮,完成位置微调。
乐白机器人采用Z-Y-X欧拉角(EulerZYX)描述机器人末端的姿态,即先绕坐标系Z轴旋转Rz角,再绕旋转后的坐标系Y轴旋转Ry角,最后绕旋转两次后的坐标系X轴旋转Rx角。
选择圆柱坐标系,左侧调整盘会由调整(X, Y)变为调整(ρ, θ)。圆柱坐标系的X-Y平面为极坐标系,其中 $\rho=\sqrt{x^2+y^2}$,$\theta=\arctan\frac{y}{x}$。
参考坐标系选择“相对于末端”,就是以机器人法兰盘平面的圆心作为坐标系原点,通过直角坐标系为调整方式,通过在微调页面输入位置或姿态的数值或拉动屏幕下方的拉杆操作,机器人自动移动到指定位置后,点击“更新”,完成位置微调。
	关节空间微调
点击微调页面左上方“关节空间”,通过输入关节1至关节6的角度数值,机器人会自动到达指定位置;或通过拉动屏幕下方各个方向的拉杆进行操作,当释放拉杆时,微调停止,点击“更新”,完成位置微调。

 
图 4 10 关节空间微调
	速度与加速时间
速度与加速时间是针对当前场景或单个位置块的速度(v)和加速度(a)进行修改,其中横轴对应加速时间,横轴数值越靠近原点,加速时间越短,加速度越大。纵轴对应速度,纵轴越靠近原点,速度越小。

 
图 4 11  速度与加速时间
速度与加速时间的调整可在场景编辑器的工具栏上全局操作(见4.3.2.8.4速度和加速时间),也可对每个位置块单独操作。 
点击位置块速度与加速时间按钮弹出调整控件,右上角的图标   为当前位置块与编辑器工具栏上的全局速度和加速时间信息的同步锁。当场景中添加新的位置时,该位置块的速度加速时间控件全局锁默认为锁定状态,即表示与当前场景的全局速度和加速时间保持一致;当用户拖拽调整控件的横轴或纵轴任意一个拖拽点,使速度与加速时间发生变化时,全局锁解锁,即该位置块使用自己指定的速度和加速时间,不再使用全局的速度和加速时间参数值;用户可以再次点击解锁图标锁定同步锁,则该位置块又保持与全局速度与加速时间一致的参数值。
平滑功能,主要是指存在连续相邻多个位置块时,机器人控制系统自动优化出最佳路径,不停止地经过位置块的目标位置,使得动作连续性更好,移动的时间更短,效率更高。具体表现为当打开平滑功能之后,两个动作块之间不会出现明显的机器人位置间移动时的减速停止现象。
在添加完位置动作块后,如果不需要添加其他类型的动作块,则可以直接查看4.3.2.8.8运行场景的内容。
手爪动作块
在添加动作块对话框中选择“手爪”,输入手爪动作描述,设置力度和幅度。如果需要预览手爪的开闭效果,请打开“预览”开关(实时预览)或点击“预览”按钮(手动预览),确认效果无误后,点击“添加”。

 
图 4 12  添加手爪动作对话框
等待动作块
在添加动作块对话框中选择“等待”,输入等待时长(单位:秒)。
消息提示动作块
在添加动作块对话框中选择“消息提示”,消息提示分为灯板提示和弹框提示。其中灯板提示可选择多种样式:
关闭:关闭灯板显示;
常亮:灯板保持指定颜色常亮;
呼吸:灯板按照指定颜色呼吸;
均分旋转:按照指定的2或4个不同颜色平均分布旋转展示;
同色旋转:按照某个颜色旋转展示;
闪烁:按照某个颜色闪烁。
数字I/O动作块
在添加动作块对话框中选择“数字I/O”,在弹出的添加对话框中,顶部Tab对应数字I/O的操作类型,数字I/O动作块支持三种类型的操作:
读取:读取某个数字I/O端口的输入值;
等待:当执行到该动作块时,将等待某个数字I/O的值为选定的值,在未变为选定的值之前,将一直停留在该动作块;
设置:设置某个数字I/O端口的输出值。
点击“添加”。
每个操作类型的Tab下,根据数字I/O端口来源分类,分为:
控制箱I/O
法兰盘I/O

 	注意:
运行任务前请确认数字I/O的输入输出电气连接正常。

模拟I/O动作块
在添加动作块对话框中选择“模拟I/O”,在弹出的添加对话框中,顶部Tab对应模拟I/O的操作类型,模拟I/O动作块支持三种类型的操作:
读取:读取某个模拟I/O端口的输入值;
等待:当执行到该动作块时,将等待某个模拟I/O的值与选定值的判断条件成立,在条件未成立之前,将一直停留在该动作块;
设置:设置某个模拟I/O端口的输出值。
点击“添加”。
模拟I/O端口来源,仅有控制箱I/O。
其中,在等待操作类型下,判断条件有:>,=,<三种,当点击某个等待模拟I/O位置块的等待条件按钮(如图 4 13 模拟I/O判断条件)时,进行相应的判断条件修改。
 
图 4 13 模拟I/O判断条件

 	注意:
运行任务前请确认模拟I/O的输入输出电气连接正常。

负载配置动作块
在添加动作块对话框中选择“负载配置”,在添加负载配置对话框中输入需要修改的负载质量或质心,该功能用于在程序运行过程中动态修改负载的质量和质心。
 	警告:
当机器人末端安装有末端工具,且工具具有可以抓取、吸附、取放物品等功能时,需要在时间轴编辑器的对应位置插入经取、放后末端负载质量和质心对应变化的动作块,如未正常设置,可能降低机器人相应部件寿命。
 	警告:
添加负载配置时,质量和质心的设置必须尽量与末端工具质量一致,更换或卸除末端工具时,一定要相应地修改负载参数或禁用对应的末端设备,否则有可能导致误伤。 
工具栏
时间轴编辑器的工具栏为查找、编辑、操作和执行该场景的工具区域:
 
图 4 14 时间轴编辑器工具栏
修改场景名称
点击工具栏标题文字后面的笔状修改按钮,在弹出的对话框中修改场景名称。
搜索
点击搜索按钮,可展开动作块搜索框。
 
图 4 15 展开的动作块搜索框
在搜索框展开状态下,点击文本输入框的 动作块类型筛选图标,可以根据弹出框中的动作块类型来进行快速搜索。同时,也可结合文本输入框中输入的关键字来查询对应类型或全部类型下的符合该关键字的动作块。
信号量
信号量在目前版本的L MASTER系统中暂无有效使用,此处略过讨论。
速度和加速时间
工具栏上的速度和加速时间为当前场景全局的速度和加速时间配置入口,当未使用4.3.2.1.2编辑位置提到的单个位置动作块的速度和加速时间调整操作时,所有该场景下的位置动作块使用全局的速度和加速时间配置。
撤销和重做
当误删除某个动作块或者执行了错误操作时,可以点击撤销按钮 执行撤销动作;反之,如果想重复之前被撤销的动作,可以点击重做按钮 执行重做动作。
 	警告:
当从当前编辑器页面返回或退出时,编辑器的撤销和重做历史将被清空,再次返回当前场景编辑器时,将无法执行之前的撤销和重做动作。

删除/清空
当编辑区选中了部分动作块时,删除/清空按钮上将展示对应选择的动作块数量,此时点击该按钮执行删除对应动作块的操作;当编辑区未选中任何动作块时,此时点击该按钮执行清空当前场景操作。
 	警告:
请务必确保您知晓执行此操作的后果并在二次确认删除或清空时,执行此操作,特别是当执行完删除或清空操作退出了当前编辑器,再次进入该场景编辑操作时,将无法还原删除或清空前的状态。

场景循环次数
点击时间轴编辑器工具栏右上角循环次数图标 修改任务循环次数(默认循环为1次);当次数为0时,表示执行无限循环(加图标)任务。
运行场景
运行场景有如下两种方法,可任选其一:
点击工具栏右上角运行任务图标 ;
双击机器人肩部按钮,如图 4 16 肩部按钮示意图所示(肩部灯板中间有“白”(补充logo中的白图标)标识的按钮)。

 
图 4 16 肩部按钮示意图
(设计制作)

 	注意:
机器人当前位置与待执行场景的第一个待运行位置 存在差异时,运行场景前会进行机器人位置安全检查。

机器人位置安全检查:
当机器人当前位置与场景第一个待运行位置一致时,场景运行,不执行位置安全检查;当机器人当前位置与场景第一个待运行位置不一致时,运行场景前,会执行位置安全检查并弹出位置安全检查页:
(补充位置安全检查页图)
 
图 4 17 位置安全检查页
在位置安全检查页可以使用如下两种操作将机器人移动到场景的第一个待运行位置:
点击位置安全检查页“移动到目标位置”,等待机器人运行至场景的第一个待运行位置,运行过程中可以随时点击停止来停止移动。
长按末端平按钮,机器人移动到任务第一个待运行位置后,放开末端平按钮。
点击右上角“完成”按钮,场景开始运行。在场景运行时,长按肩部按钮可以暂停或恢复任务。
 	警告:
在使用末端平按钮移动到第一个待运行位置后,请与机器人保持一定安全距离,点击“完成”按钮去运行场景,否则可能会造成误伤。
操作技巧
快速搜索动作块
将光标放在选中的动作块上使其处于焦点状态,点击鼠标右键,在弹出的右键菜单中选择“查找相似块”,可快速搜索出相同动作块类型且包含当前选中的动作块标题作为关键字的动作块。
批量编辑动作块
通过搜索框或快速搜索动作块等操作选中2个或以上的同类型且该类型动作块内的子类型(如果该类型动作块存在子类型的话)相同的动作块,点击编辑区左上角工具栏的“批量修改…”按钮,或者将光标放在选中的动作块上使当前动作块处于焦点状态,点击鼠标右键,选择“批量修改…”,可对动作块以下内容进行批量编辑:
位置动作块:统一修改位置名称、切换关节空间的运动(movej)和笛卡尔空间的直线运动(movel),速度及加速时间以及启用和禁用平滑功能,且批量修改确认后每个位置块的位置数据将应用机器人当前状态下的位置数据。
手爪动作块:手爪动作描述,力度以及幅度;
等待动作块:等待秒数以及等待的目的/作用描述;
消息提示动作块:仅支持同一个子类型的消息提示:灯板提示或弹窗提示的批量修改,不同子类型的消息提示不可批量修改;
	数字I/O动作块:仅支持同子类型的I/O操作:读取或等待或设置的批量修改,不同子类型不可批量修改;
	模拟I/O动作块:仅支持同子类型的I/O操作:读取或等待或设置的批量修改,不同子类型不可批量修改;
	负载配置动作块:仅支持同子类型的负载质量或质心的批量修改,不同子类型不可批量修改。

点击对话框右下角的“修改”按钮可应用批量编辑操作,点击右上角的“X”按钮可取消批量编辑操作。

 	注意:
动作块批量编辑,动作块类型和子类型(如果该类型动作块存在子类型的话)必须一致。


批量微调位置块
通过搜索框或快速搜索动作块等操作选中2个或以上的位置动作块,点击编辑区的“批量微调位置…”,或者将光标放在选中的位置动作块上使当前位置动作块处于焦点状态,点击鼠标右键,选择“批量微调位置…”,进入“调整位置”页面,具体操作详见4.3.2.1.2的(1)微调。微调移动至目标位置后,点击页面右上角的“更新”按钮,完成批量微调;点击“取消”可放弃批量微调操作。其中,“更新”按钮右侧数字显示的是当前选中需要批量微调位置块的总数。
 
图 4 18 批量微调位置块
导出场景
在场景列表页面选择需要保存的场景,点击 按钮,选择“导出”,在弹出的保存对话框选择场景文件的保存位置。 
 
 
图 4 19 导出场景
导入场景
在场景列表页面工具栏点击“导入场景”按钮,打开需要导入的场景文件(以lbd结尾的文件),导入完成后会自动进入该场景的编辑页面。
 
图 4 20 导入场景
控制
控制模块主要分为:
虚拟控制
位置库
I/O控制
手爪控制
硬件按钮
灯板控制

虚拟控制
通过虚拟控制“坐标空间”和“关节空间”可以调整机器人当前位置与姿态,具体操作方式参考4.3.2.1.2编辑位置中的微调功能介绍。
 
图 4 21  虚拟控制示意图
硬件按钮
	末端平按钮
单击:时间轴编辑焦点往后移动一个动作块 ;
双击:时间轴编辑焦点往前移动一个动作块 ;
长按:当前如果处于位置动作块刷新位置弹框(点击位置块的微调按钮进入)或者位置库应用位置弹框时,则移动到对应的目标位置;
释放:当长按进入移动到目标位置操作时,释放将停止当前移动。
 
 
图 4 22  末端按钮示意图
(设计修改:block改为位置块,平按钮图示错误,平按钮和凸按钮添加文字标注)

	末端凸按钮
长按:进入示教模式;
释放:退出示教模式;
双击: 在编辑器中添加一个位置动作块 /代码;
单击: 当前动作块为位置动作块的前提下 ,如果当前位置动作块未进入微调位置对话框,单击该按钮进入微调当前位置对话框; 如果当前已进入微调位置对话框,单击该按钮则表示更新当前动作块保存的位置数据。

	肩部按钮
长按: 切换队列暂停/恢复操作,即运行中时暂停,暂停时恢复;
单击: 场景编辑器有弹框时或者其他界面:等价于点击取消操作(按钮,对话框等);  
双击: 当前在场景编辑器界面且无其他弹框:则切换运行/停止操作 ;场景编辑器有弹框时或者其他界面:等价于点击确认操作(按钮,对话框等)。 

	按钮组合操作 
启动/停止机器人:长按末端平按钮,同时长按肩部按钮,可以切换启动和停止机器人的操作(仅在当前机器人未急停或未断电时有效)。

设备
末端设备
如果需要在机器人末端添加末端工具(如:手爪),点击“设备”,在末端设备选择“添加设备”,设置相应的辅助工具的质量和质心,点击“启用”。如果需要卸除末端工具,点击“禁用”。
机器人最大允许有效负载取决于重心偏移,如图 4 23 有效负载图。重心偏移定义为工具输出法兰的中心与重心之间的距离。
 
图 4 23 有效负载图
 	警告 :
负载条件应在图表所示范围内;
图中显示的有效负载表示的是最大负载能力,在任何情况下,都不应该超过图中所示的最大重量;
超过允许值会导致机器内部件的提早损坏。
 	警告:
添加末端设备时,末端设备的质量和质心的设置必须与末端工具质量和质心尽可能一致;
更换或卸下末端工具时,一定要相应地修改末端设备的质量和质心或关闭对应末端设备,否则有可能导致误伤。
设置
 TCP设置
TCP设置用于设置机器人末端TCP位置和姿态的偏移量/转换量,默认不设置TCP,当机器人装载末端工具时可以根据场景应用需要选择性添加TCP设置。
TCP可以通过示教添加。点击“添加”按钮菜单的“示教添加”,打开“示教添加TCP设置”对话框,如图5-19。此时可点击示教图标,拖动示教机器人,使得机器人末端工具在接触同一控制点(即保持末端工具始终接触同一个位置)的情况下,以四个不同姿态逐一确认四个关键点,机器人根据末端法兰不同的位置和方向,可自动识别TCP位置信息。姿态信息需要手动填写,暂不支持示教添加。
 
图 4 24 示教添加TCP设置
用户还可以通过“手动添加”或者“编辑”,进入“编辑TCP设置”对话框,如图5-20,手动设置末端工具的位置和姿态表示。
 
图 4 25  编辑TCP设置

安全设置
碰撞检测
机器人碰撞检测在安装引导页开启后默认状态为“急停”。机器人在运行任务过程中,检测到外部阻力的碰撞后的动作分为暂停和急停两种,同时用户可以自行调节检测的灵敏度。

  
图 4 26 碰撞检测设置

碰撞后的动作 
急停:需要重新启动机器人,示教至安全位置后才能继续操作;
暂停:若选择自定义秒数,到达指定暂停时间时,任务会自动恢复运行;若选择永久暂停,需要在首页任务历史列表栏点击恢复任务按钮 。

运行安全
在运行安全模块中,可以查看和编辑如下运行时的参数:
每个关节的最大角度和最小角度限制
在关节空间运行时的最大速度和最大加速度限制
在坐标空间运行时的最大速度和最大加速度限制
其中,关节空间运行时的最大速度和最大加速度限制应用于场景编辑在时间轴编辑器模式下,使用滑杆调整速度和加速时间(加速度)时的最大值,控制系统限制关节空间相关移动(movej)的速度和加速度的最大值以及全局限定每个关节运行时的最大速度限制时有效。当任意一个关节运行时的速度超过最大速度限制时,机器人会自动急停。
坐标空间运行时的最大速度和最大加速度限制应用于场景编辑器在时间轴编辑器下,使用滑杆调整速度加速时间(加速度)时的最大值以及控制系统限制坐标空间相关移动(movel, movec)的速度和加速度的最大值。

 
图 4 27运行安全设置

 	警告:
非专业用户在不确定修改后的风险情况下,不可随意更改。

操作模式
操作模式分为如下两种,可根据实际情况选择:
新手模式
新手模式使用低门槛时间轴编辑器。在新手模式下,专家模式的部分高级功能将被隐藏,以尽可能减少复杂和高级功能对新手使用的困扰。
专家模式
专家模式使用专业代码编辑器。在专家模式下:
可支持切换编辑器类型为代码编辑器(代码编辑器使用Lua语言进行编辑);
支持将时间轴编辑器版本的场景转换成Lua代码 ;
支持时间轴编辑器模式下关闭位置安全检查(请务必确保您已足够了解该操作的危险性后再执行关闭操作);
支持安装设置可自定义任意安装方式的配置。

操作模式的切换:
在“设置页面”的“操作模式”中切换并保存;
场景编辑右上角的切换图标,切换并保存。
系统更新
如果当前系统已是最新版本,如图 4 28 系统更新页面-已是最新版本画面中心图标显示蓝色,点击“检查系统”,画面中心圆形图案显示“正在检查更新”。
 	注意:
请确认机器人状态为停止后,进行系统更新操作;
系统更新中,不可进行其他操作,否则可能导致机器人系统损坏等严重后果。
 
图 4 28 系统更新页面-已是最新版本(设计去除左侧绿色标识)
当检测到系统版本有更新时,如图 4 29 系统更新页面-检测到更新,画面中心图标显示橙色,并提示“发现系统更新及最新版本号”。 点击下方“更新”,系统自动更新至最新版本。
 
图 4 29 系统更新页面-检测到更新(设计去除左侧绿色标识)
